%% WARNING: THIS IS A DERIVED (FLATTENED) FILE; DO NOT MAKE
%% SUBSTANTIVE EDITS! EDIT THE ORIGINAL SOURCE FILES INSTEAD.

% flatex input: [merkle.tex]
\documentclass[11pt,hidelinks]{article}
\usepackage{fullpage}
\usepackage{microtype}
\usepackage{newtxtext}
\usepackage{xspace}

\usepackage{amsmath,amsfonts,amssymb}
\usepackage[amsmath,amsthm,thmmarks,hyperref]{ntheorem}
\usepackage{mathtools}
% https://tex.stackexchange.com/questions/424686/crefrange-links-to-the-right-line-in-the-wrong-algorithm
\usepackage[colorlinks,hypertexnames=false]{hyperref}
\hypersetup{linkcolor=blue,filecolor=blue,citecolor=blue,urlcolor=blue}

\newtheorem{theorem}{Theorem}[section]

\usepackage{algorithm}
\usepackage[noend]{algpseudocode}
\renewcommand{\algorithmicrequire}{\textbf{Input:}}
\renewcommand{\algorithmicensure}{\textbf{Output:}}

\algnewcommand{\IfThenElse}[3]{% \IfThenElse{<if>}{<then>}{<else>}
  \State \algorithmicif\ #1\ \algorithmicthen\ #2\ \algorithmicelse\ #3}

\usepackage{enumitem}
\usepackage{breakcites}
\usepackage{authblk}

% load after hyperref, algpseudocode to get proper behavior
\usepackage[capitalize,nameinlink,noabbrev]{cleveref}

% remove 'draft' to turn off fixme notes
\usepackage[draft,multiuser,inline,nomargin]{fixme}

% fixme: register commands for author(s)
\usepackage{xcolor}
\FXRegisterAuthor{c}{ec}{\color{red}Chris}
\FXRegisterAuthor{r}{er}{\color{purple}Riad}
\fxusetheme{color}

% flatex input: [macros.tex]
\newcommand{\algo}[1]{\ensuremath{\mathsf{#1}}}

\newcommand{\bit}{\ensuremath{\{0,1\}}}
\newcommand{\msgspace}{\ensuremath{\mathcal{M}}}
\newcommand{\hashsp}{\ensuremath{\mathcal{H}}}
\newcommand{\specialhash}{\ensuremath{\mathbf{0}}}

\newcommand{\MT}{\algo{MT}}
\newcommand{\VC}{\algo{VC}}

\newcommand{\Build}{\algo{Build}}
\newcommand{\Commit}{\algo{Commit}}
\newcommand{\Prove}{\algo{Prove}}
\newcommand{\Verify}{\algo{Verify}}

\newcommand{\HashEntry}{\algo{HashE}}
\newcommand{\HashChildren}{\algo{HashC}}
\newcommand{\RootToLeafPath}{\algo{RootToLeafPath}}

\newcommand{\Sig}{\algo{Sig}}
\newcommand{\Gen}{\algo{Gen}}
\newcommand{\Sign}{\algo{Sign}}

\newcommand{\MSS}{\algo{MSS}}

% flatex input end: [macros.tex]

% fixme: register commands for author(s)

\title{Merkle Trees and Signatures}

% date of last substantive edits
\date{February 8, 2022}

\author{Chris Peikert}
\author{Idan Meshita}
\author{Riad Wahby}

\affil{Algorand, Inc.}

\begin{document}

\maketitle

% list of notes to address; appears only in draft mode
\listoffixmes

% flatex input: [trees.tex]

For a non-negative integer~$k$, let $[k] := \{0, 1, \ldots,
k-1\}$. Recall that a \emph{collision} in a function~$H$ is a pair of
distinct inputs $m \neq m'$ such that $H(m) = H(m')$, and a
\emph{claw} in a pair of functions $H,H'$ is a pair of (not
necessarily distinct) inputs $m,m'$ such that $H(m) = H'(m')$.

\section{Merkle Trees and Vector Commitments}
\label{sec:merkle-vc}

This section defines:
\begin{enumerate}[itemsep=0pt]
\item a version of Merkle trees, which allows for concisely committing
  to a possibly large, implicitly \emph{unordered} set of data
  entries;
\item a version of vector commitments, which allows for concisely
  committing to an \emph{ordered} (indexed) vector of data entries,
  based on Merkle trees.
\end{enumerate}

\paragraph{Binary trees.}

A \emph{complete} (also known as \emph{almost complete}) binary tree
is one in which every level is completely filled, except for possibly
the leaf level, in which the leaves are filled from the left. Such a
tree~$T$ can be:
\begin{itemize}[itemsep=0pt]
\item \emph{empty}, meaning it has zero nodes (not even a root node);
\item a \emph{leaf}, meaning it has exactly one node (which is its
  root);
\item or neither of these, in which case it has a non-empty left child
  tree denoted $T.\text{left}$, and a (possibly empty) right child
  tree denoted $T.\text{right}$.
\end{itemize}
A \emph{perfect} binary tree is a complete one where every layer,
including the leaf layer, is completely filled.

\paragraph{Merkle tree overview.}

In the present version of Merkle trees, the data entries are placed at
the leaves of an (almost) complete binary tree, with one entry per
leaf. Every binary tree~$T$, even the empty one, has an attribute
$T.\text{digest}$, which holds a hash digest representing the contents
of the entire tree:
\begin{itemize}[itemsep=0pt]
\item The digest of the empty tree is always defined to be
  $T.\text{digest} = \specialhash$, a canonical (but arbitrary)
  digest value.
\item The digest of a leaf tree is defined to be the hash of its
  associated data entry, using a specified function for hashing
  entries.
\item The digest of a non-empty, non-leaf tree is defined to be the
  hash of the digests of its left and right child trees, using a
  specified function for hashing pairs of digests.
\end{itemize}
The two hash functions should satisfy certain collision-resistance
properties individually and together; in particular, they must be
different functions, but they can be defined using a single
collision-resistant hash function via appropriate domain
separation. See \cref{alg:merkle} and \cref{thm:tree-vc} for further
details.

\medskip
\noindent The abstract syntax (interface) for Merkle trees is as
follows (see \cref{alg:merkle} for the formal definitions):
\begin{itemize}
\item $\MT.\Build$ takes an (almost) complete tree with a data entry
  at each one of its leaves, and populates the digests of the tree and
  all its subtrees, also returning the digest of the entire tree.

\item $\MT.\Prove$ takes a built tree (i.e., one whose digests have
  been populated by $\MT.\Build$) and a root-to-leaf path specified by
  a binary string (where~$0$ and~$1$ respectively denote left and
  right), and returns a proof consisting of one hash digest per bit of
  the string. If the provided path does not lead from the root to a
  leaf (because the path is either too short or too long), then the
  output is an error. In particular, there is no valid path for an
  empty tree.

\item $\MT.\Verify$ takes a digest representing a commitment to an
  entire tree (of arbitrary size and topology), a claimed data entry,
  a root-to-leaf path specified by a binary string of some
  non-negative length (possibly zero), and a purported proof
  consisting of one hash digest per bit of the string. It either
  accepts or rejects, indicating whether these inputs represent a
  valid proof that the claimed entry is located at the leaf specified
  by the claimed path.
\end{itemize}

\paragraph{Vector commitment overview.}

The abstract syntax (interface) for vector commitments is as follows
(see \cref{alg:vc-merkle} for the formal definitions):
\begin{itemize}
\item $\VC.\Commit$ takes an ordered vector of any non-negative number
  (possibly zero) of data entries, and returns a commitment along with
  some auxiliary data that is later used for proving individual
  entries.
  
\item $\VC.\Prove$ takes the auxiliary data (as returned by
  $\VC.\Commit$) and a valid index into the committed vector, and
  returns a proof.  (Concretely, the proof is an ordered tuple of hash
  digests, as returned by $\MT.\Prove$.)
  
\item $\VC.\Verify$ takes a commitment (concretely, a hash digest) to
  some vector of any valid dimension, a vector index, a claimed data
  entry, and a purported proof (concretely, an ordered tuple of some
  non-negative number of hash digests). It either accepts or rejects,
  indicating whether these inputs represent a valid proof that the
  claimed entry is located at the specified index of the committed
  vector. The provided index must be in $[2^{d}]$, where~$d$ is the
  number of hash digests in the proof; otherwise $\VC.\Verify$
  immediately rejects.
\end{itemize}

The vector commitment scheme $\VC$ is defined essentially as a ``thin
wrapper'' around the Merkle tree scheme $\MT$, with two pieces
involved in the translation. First, $\VC$ uses a suitable
correspondence between vector indices and root-to-leaf paths. Namely,
index~$i$ corresponds to a path specified by the binary representation
of~$i$, from \emph{least}- to \emph{most}-significant bit. This
ordering (which is the ``bit-reversed'' version of the usual indexing
of leaves in a binary tree) is \emph{essential} for the
position-binding security property; see \cref{thm:tree-vc} and the
discussion following it. Note that a given path uniquely defines an
index (e.g., path $0111$ corresponds to index~$14$), but a given index
can yield multiple paths (e.g., index~$13$ corresponds to paths
$1011$, $10110$, $101100$, etc.). However, the $\VC$ functions always
have additional context that determines the proper path length to use
(called~$d$ in the pseudocode), and with this context, an index
uniquely defines a path.

Second, to ensure that $\VC$ provides $\MT$ with a complete tree
(which is not immediate, due to the above-described indexing),
$\VC.\Commit$ also pads its input vector with special, distinguished
`nothing' values to make the vector have a power-of-two length. (These
`nothing' values must be hashed appropriately, which can be done by
suitable domain separation.) This ensures that the resulting tree is
perfect, and hence complete.

\begin{algorithm}
  \caption{Merkle tree scheme ($\MT$).}
  \label{alg:merkle}

  \begin{itemize}[itemsep=0pt]
  \item Let $\msgspace = \bit^{*} \cup \{ \bot \}$, the set of all
    binary strings together with a distinguished `nothing'
    value~$\bot$, denote the space of valid vector entries.
  \item Let $\HashEntry \colon \msgspace \to \hashsp$ be a hash
    function for vector entries, and let
    $\HashChildren \colon \hashsp \times \hashsp \to \hashsp$ be a
    hash function for children of internal nodes. Let
    $\specialhash \in \hashsp$ denote a special hash digest, which may
    be arbitrary.

    These hash functions can be instantiated using a single function
    with appropriate domain separation.
  \end{itemize}

  \begin{algorithmic}[1]
    \Function{\MT.\Build}{$T$}

    \Comment{in a \emph{complete} binary tree, compute all digests
      from its leaf entries (if any); return the tree's digest}

    \If{$T$ is empty} do nothing
    \Comment{$T.\text{digest} = \specialhash$ already}
    \ElsIf{$T$ is a leaf}
    $T.\text{digest} = \HashEntry(T.\text{entry})$
    \Else\ $T.\text{digest} =
    \HashChildren(\MT.\Build(T.\text{left}), \MT.\Build(T.\text{right}))$
    \EndIf

    \State \Return $T.\text{digest}$
    \EndFunction

    \Function{\MT.\Prove}{$T, p = p_{1}\cdots p_{d} \in \bit^{d}$}
    \Comment{prove entry at the root-to-leaf path~$p$ of a built tree~$T$}

    \If{$d = 0$} \Comment{path is empty\ldots}
    \IfThenElse{$T$ is a leaf}{\Return $()$}{\Return error}
    \Comment{\ldots so tree must be a leaf}
    \EndIf
    
    \If{$T$ is empty or a leaf}
    \Return error \Comment{path is non-empty, so children are required}
    \EndIf

    \IfThenElse{$p_{1} = 0$}
    {$T' = T.\text{left}, \pi=T.\text{right}.\text{digest}$}
    {$T' = T.\text{right}, \pi=T.\text{left}.\text{digest}$}
    
    \State \Return $\pi, \MT.\Prove(T', p_{2}\cdots p_{d})$
    \Comment{prepend sibling digest to proof for subtree}
    \EndFunction

    \Function{\MT.\Verify}{$C \in \hashsp, m \in \msgspace, p = p_{1}
      \cdots p_{d} \in \bit^{d}, \pi = (\pi_{1}, \ldots, \pi_{d}) \in
      \hashsp^{d}$}

    \Comment{for commitment~$C$, verify a proof~$\pi$ that the entry
      at root-to-leaf path~$p$ (both of length $d \geq 0$) is~$m$}

    \State let $h_{d} = \HashEntry(m) \in \hashsp$

    \For{$j = d, \ldots, 1$} \Comment{hash from leaf to root (doing
      nothing if $d=0$)}
    \IfThenElse{$p_{j} = 0$}
    {$h_{j-1} = \HashChildren(h_{j}, \pi_{j})$}
    {$h_{j-1} = \HashChildren(\pi_{j}, h_{j})$}
    \Comment{hash left then right}
    \EndFor

    \IfThenElse{$h_{0} = C$}{\Return accept}{\Return reject}
    
    \EndFunction
  \end{algorithmic}
\end{algorithm}

\begin{algorithm}
  \caption{Vector-commitment scheme ($\VC$) based on Merkle trees.}
  \label{alg:vc-merkle}
  
  \begin{algorithmic}[1]
    \Function{\VC.\Commit}{$m_{0}, m_{1}, \ldots, m_{k-1} \in \msgspace$}
    \Comment{commit to a vector of any dimension $k \geq 0$}

    \State let $d \geq 0$ be the smallest integer such that $2^{d}
    \geq k$

    \For{$i = k, \ldots, 2^{d}-1$} let $m_{i} = \bot$ \EndFor

    \State construct a perfect depth-$d$ binary tree~$T$, where:

    for each $i \in [2^{d}]$, the entry given by the root-to-leaf path
    $\RootToLeafPath(d, i)$ is~$m_{i}$.

    \State $C = \MT.\Build(T)$

    \State \Return commitment~$C$ and auxiliary proving data $(T,k,d)$

    \EndFunction

    \Function{\VC.\Prove}{$(T,k,d), i \in [k]$}
    \Comment{open entry~$i$ of a commitment using auxiliary data $(T,k,d)$}

    \State \Return $\MT.\Prove(T, \RootToLeafPath(d, i))$

    \EndFunction

    \Function{\VC.\Verify}{$C \in \hashsp, i \in [2^{d}], m \in
      \msgspace; \pi = (\pi_{1}, \ldots, \pi_{d}) \in \hashsp^{d}$}

    \Comment{for commitment~$C$, verify a proof~$\pi$ that entry~$i$
      is~$m$ (where $d$ is determined by~$\pi$)}

    \If{$i \not\in [2^{d}]$} \Return false
    \Comment{explicitly reject out-of-bounds indices}
    \EndIf

    \State \Return $\MT.\Verify(C, m, \RootToLeafPath(d, i), \pi)$
    \Comment{use the path to the leaf storing entry~$i$}

    \EndFunction

    \Function{\RootToLeafPath}{$d \geq 0, i \in [2^{d}]$} \Comment{the
      root-to-leaf path $p \in \bit^{d}$ for index~$i$, LSB to MSB}

    \IfThenElse{$d=0$}{\Return $()$}
    {\Return $(i \bmod 2,
      \RootToLeafPath(d-1, \lfloor i/2 \rfloor) \in \bit^{d})$}

    \EndFunction
  \end{algorithmic}
\end{algorithm}

\begin{theorem}
  \label{thm:tree-vc}
  If both~$\HashEntry$ and $\HashChildren$ are collision resistant,
  and the pair $\HashEntry,\HashChildren$ is claw resistant, then the
  vector-commitment scheme from \cref{alg:vc-merkle} is position
  binding.

  That is, it is infeasible for an adversary to output a
  commitment~$C$, an index~$i$, two \emph{distinct} messages
  $m,m' \in \msgspace$, and two proofs $\pi, \pi'$ (of any, possibly
  different lengths) such that both $\VC.\Verify(C,i,m,\pi)$ and
  $\VC.\Verify(C,i,m',\pi')$ accept.
\end{theorem}

Note that this theorem crucially relies on the fact that the
root-to-leaf path to entry~$i$ follows the \emph{LSB-to-MSB} binary
representation of~$i$. By contrast, using the MSB-to-LSB
representation (as is done in some Merkle tree implementations)
\emph{makes it trivially easy to break the position binding
  property}. (For example, an adversary can easily create a commitment
and open its entry $i=1$ in two different ways, using proofs of
different `depths'~$d$.) Therefore, for security it is \emph{vital} to
use a proper correspondence between indices and paths.

\begin{proof}
  % First observe that $\HashEntry \colon \msgspace \to \hashsp$ is
  % collision resistant, because any given collision in $\HashEntry$
  % yields a collision in~$H$ due to the different prefixes used for
  % `nothing' and non-`nothing' inputs. Similarly, the pair
  % $\HashEntry, \HashChildren$ is claw resistant, because any given
  % claw for those functions immediately yields a claw in
  % $H, \HashChildren$.

  Consider an attacker against the position binding of $\VC$. Suppose
  that it successfully outputs some
  $C \in \hashsp, i \in [2^{d}], m \in \msgspace, \pi=(\pi_{1},
  \ldots, \pi_{d}) \in \hashsp^{d}, m' \in \msgspace, \pi'=(\pi'_{1},
  \ldots, \pi'_{d'}) \in \hashsp^{d'}$ where $d \leq d'$ without loss
  of generality, $m \neq m'$, and both $\VC.\Verify(C, i, m, \pi)$ and
  $\VC.\Verify(C, i, m', \pi')$ accept. From such an output we show
  how to efficiently extract a collision in $\HashEntry$ or
  $\HashChildren$, or a claw in $\HashEntry,\HashChildren$.

  Let
  \begin{align*}
    p &= p_{1} p_{1} \cdots p_{d} = \RootToLeafPath(d,i) \in \bit^{d} \\
    p' &= p'_{1} p'_{1} \cdots p'_{d} = \RootToLeafPath(d',i) \in \bit^{d'}
         \text{.}
  \end{align*}
  By definition of \RootToLeafPath, $p$ is a prefix of~$p'$, i.e.,
  $p_{j}=p'_{j}$ for all $j \in [d]$. Note that this is where we use
  the fact that $\RootToLeafPath$ outputs the LSB-to-MSB binary
  representation of~$i$. (Under the MSB-to-LSB representation, this
  prefix property would not hold.)

  Let digests $h_{j}, h'_{j} \in \hashsp$ be as in the execution of
  $\MT.\Verify(C, m, p, \pi)$ and $\Verify(C, m', p', \pi')$,
  respectively.  By definition, we have $h_{0} = C = h'_{0}$.
  Let~$j \in \{1, 2, \ldots, d\}$ be the smallest integer such that
  $(h_{j},\pi_{j}) \neq (h'_{j},\pi'_{j})$, if such an integer
  exists. We proceed with a case analysis based on whether~$j$ exists
  and whether $d=d'$.

  First suppose that~$j$ exists. Then if $p_{j}=p'_{j}=0$, we have
  \[ \HashChildren(h_{j}, \pi_{j}) = h_{j-1} = h'_{j-1} =
    \HashChildren(h'_{j}, \pi'_{j}) \text{,} \] so
  $(h_{j},\pi_{j}) \neq (h'_{j}, \pi'_{j})$ is a collision
  in~$\HashChildren$, as needed. The same reasoning holds if
  $p_{j}=p'_{j}=1$, but with the arguments to~$\HashChildren$ swapped.

  If no such~$j$ exists, then $h_{d}=h'_{d}$. If $d=d'$, then we have
  \[ \HashEntry(m) = h_{d} = h'_{d} = \HashEntry(m') \text{,} \] so
  $m \neq m'$ is a collision in $\HashEntry$, as needed. Otherwise
  $d < d'$, and if $p'_{d+1}=0$ we have
  \[ \HashEntry(m) = h_{d} = h'_{d} = \HashChildren(h'_{d+1},
    \pi'_{d+1}) \text{,} \] so $m, (h'_{d+1},\pi'_{d+1})$ is a claw in
  the pair $\HashEntry, \HashChildren$, as needed. The same reasoning
  holds if $p'_{d}=1$, but with the arguments to~$\HashChildren$
  swapped. This completes the proof.
\end{proof}

% flatex input end: [trees.tex]

% list of notes to address; appears only in draft mode

% flatex input: [sigs.tex]

\section{Merkle Signatures}
\label{sec:merkle-signatures}

This section specifies the \emph{Merkle signature scheme} (MSS), which
generically combines any vector-commitment scheme and a signature
scheme $\Sig = (\Gen, \Sign, \Verify)$ to get another signature
scheme.  The key property of MSS is that it generates several
\emph{ephemeral} public and secret keys, each of which is only useful
for a limited time, and the secret keys can be deleted once they have
been used (or expire). This mitigates against the risk of a
cryptanalytic attack on a long-lived public key.

To implement limited-time keys, MSS extends the usual syntax of a
signature scheme via the concept of a \emph{round}. The current value
of the round is maintained externally (e.g., by a clock or a
blockchain) and is publicly known. The set of rounds for which a given
MSS key can sign is chosen in advance by the signing party, at the
time of MSS key generation. The MSS signing algorithm is given the
current round value, along with (as usual) the secret key and
message. Finally, the MSS verifier is also given a specified round
along with (as usual) a public key, message and purported signature,
and checks that the signature is valid for that specific round.

\begin{algorithm}
  \caption{Merkle signature scheme ($\MSS$).}
  \label{alg:mss}

  \begin{algorithmic}[1]
    \Function{$\MSS.\Gen$}{$r_{0}, r_{1}, \ldots, r_{k-1}$}
    \Comment{generate keys for signing at each round~$r_{i}$}

    \For{$i \in [k]$} $(pk_{i}, sk_{i}) \gets \Sig.\Gen()$ \EndFor

    \State let
    $(C, aux) = \VC.\Commit((pk_{0}, r_{0}), \ldots, (pk_{k-1},
    r_{k-1}))$ \Comment{commit to all key-round pairs}

    \State \Return public key $pk_{\MSS}=C$, non-secret $aux$, and
    secret key $sk_{\MSS} = (sk_{0}, \ldots, sk_{k-1})$

    \EndFunction

    \Function{$\MSS.\Sign$}{$sk_{\MSS}, aux, r, M \in \bit^{*}$}
    \Comment{sign~$M$ in round~$r$}

    \State in $aux$, find an index~$i$ for which $r_{i}=r$ (and fail
    if no such~$i$ exists)

    \State let $\pi = \VC.\Prove(aux, i)$

    \State let $\sigma \gets \Sig.\Sign(sk_{i}, M)$

    \State \Return signature $\sigma_{\MSS} = (i, pk_{i}, \pi, \sigma)$
    \EndFunction

    \Function{$\MSS.\Verify$}
    {$pk_{\MSS}, r, M \in \bit^{*}, \sigma_{\MSS} = (i^{*}, pk^{*},
      \pi^{*}, \sigma^{*})$}

    \Comment{verify a signature on~$M$ for round~$r$}

    \State \Return
    $\VC.\Verify(pk_{\MSS}, i^{*}, (pk^{*}, r), \pi^{*})$ AND
    $\Sig.\Verify(pk^{*}, M, \sigma^{*})$

    \EndFunction
  \end{algorithmic}
\end{algorithm}

% flatex input end: [sigs.tex]

\end{document}

%%% Local Variables:
%%% mode: latex
%%% TeX-master: t
%%% End:

% flatex input end: [merkle.tex]
